%
% 33lip.tex
% 「その③ 面倒なのでピクセル単位」
%

%% 画像オブジェクト
\pgfdeclareimage[width=0.5\paperwidth]%
  {trial-lip-ok}{image/trial-lip-ok.pdf}
\pgfdeclareimage[width=0.5\paperwidth]%
  {trial-lip-ng}{image/trial-lip-ng.pdf}

\begin{mygroup}
\setbeamercolor{background canvas}{bg=mydbad}
\begin{frame}[plain]{}
  \begin{center}\LARGE\color{white}
    \mybadbkname[0.5]{③}{面倒なので\\ポイント単位}%
        {lazily-in-point}
  \end{center}
\end{frame}
\end{mygroup}

\begin{mybgcolor}{mylbad}

\begin{myframe}
  \begin{mycenter}
    \begin{tikzpicture}
      \useasboundingbox[drawbb](-5.5,-4.5) rectangle (5.5,4.5);
      \node[font=\LARGE] at (0,4)
        {\alert{大昔の}{\TeX}環境では…};
      \node[font=\large] at (0,3)
        {(2005年頃の{\TeX}環境)};
      \node[imframe] at (0,-0.3) {\pgfuseimage{trial-lip-ok}};
      \node[text=mybad, font=\hhhuge] at (-1.5,-4) {\facebad};
      \node[text=mybad, font=\hhuge] at (1.5,-4) {正常。};
    \end{tikzpicture}
  \end{mycenter}
\end{myframe}

\begin{myframe}
  \begin{mycenter}
    \begin{tikzpicture}
      \useasboundingbox[drawbb](-5.5,-4.5) rectangle (5.5,4.5);
      \node[font=\LARGE] at (0,4)
        {\alert{新しい}{\TeX}環境では…};
      \node[font=\large] at (0,3)
        {({\TeX} Live 2014)};
      \node[imframe] at (0,-0.3) {\pgfuseimage{trial-lip-ok}};
      \node[text=mybad, font=\hhhuge] at (-1.5,-4) {\facebad};
      \node[text=mybad, font=\hhuge] at (1.5,-4) {正常\!};
    \end{tikzpicture}
  \end{mycenter}
\end{myframe}

\end{mybgcolor}

\begin{mybgcolor}{mylbanned}

\begin{myframe}
  \begin{mycenter}
    \begin{tikzpicture}
      \useasboundingbox[drawbb](-5.5,-4.5) rectangle (5.5,4.5);
      \node[font=\LARGE] at (0,4)
        {\alert{少し古い}{\TeX}環境では…};
      \node[font=\large] at (0,3)
        {({\TeX} Live 2012)};
      \node[imframe] at (0,-0.3) {\pgfuseimage{trial-lip-ng}};
      \node[text=mybanned, font=\hhhuge] at (-1.5,-4) {\facebanned};
      \node[text=mybanned, font=\hhuge] at (1.5,-4) {破滅。};
    \end{tikzpicture}
  \end{mycenter}
\end{myframe}

\begin{myframe}
  \begin{center}\color{mybanned}
    {\hhhuge 楽しい…{}?\par}
    \myvspace{1}
    {\Large \mytanoshy\par}
  \end{center}
\end{myframe}

\end{mybgcolor}

\begin{mybgcolor}{mydbanned}

\begin{myframe}
  \begin{center}\color{white}
    {\hhhhuge\bfseries 楽しくない\!\!\par}
    \myvspace{1}
    {\gigantic \facebanned\par}
  \end{center}
\end{myframe}

\end{mybgcolor}

\begin{mybgcolor}{mylgood}

\begin{myframe}
  \begin{mycenter}
    \begin{tikzpicture}
      \useasboundingbox[drawbb](-5.5,-4.5) rectangle (5.5,4.5);
      \node[text=black, font=\hhuge\bfseries, xscale=0.95]
        at (0,3) {\mygood{正しい}コードを書こう!};
      \node[text=mygood, font=\gigantic] at (0,0) {\facegood};
      \node[text=mygood, font=\fntVMin\LARGE, rotate=-90]
        at (-1.9,-2.4) {まさに};
      \node[text=mygood, font=\fntVMin\hhuge, rotate=-90]
        at (-2.7,-3.0) {正論。};
    \end{tikzpicture}
  \end{mycenter}
\end{myframe}

\end{mybgcolor}

% EOF
