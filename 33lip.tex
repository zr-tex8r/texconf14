%
% 33lip.tex
% 「その③ 面倒なのでピクセル単位」
%

%% 画像オブジェクト
\pgfdeclareimage[width=0.6\paperwidth]%
  {trial-lip-ok}{image/trial-lip-ok.pdf}
\pgfdeclareimage[width=0.6\paperwidth]%
  {trial-lip-ng}{image/trial-lip-ng.pdf}

\begin{mygroup}
\setbeamercolor{background canvas}{bg=mydbad}
\begin{frame}[plain]{}
  \begin{center}\LARGE\color{white}
    \mybadbkname[0.5]{③}{面倒なので\\ポイント単位}%
        {lazily-in-point}
  \end{center}
\end{frame}
\end{mygroup}

\begin{myframe}
  \begin{mycenter}
    \begin{tikzpicture}
      \useasboundingbox[drawbb](-5.5,-4.5) rectangle (5.5,4.5);
      \node[font=\LARGE] at (0,3.5)
        {bboxの値は\myiswhat};
      \node[right, text=structure, font=\bfseries] at (-4.8,1.5)
        {image.png};
      \node[imframe] at (-3,-0.5) (Image)
        {\pgfuseimage{texconf14png}};
      \node[right, text=black, font=\large] at (-1.5,-0.5)
        {\begin{minipage}{12em}
          \begin{itemize}
          \item PNG形式
          \item 解像度 = 96\,dpi
          \item \alert{480\,px × 480\,px}
          \end{itemize}
        \end{minipage}};
    \end{tikzpicture}
  \end{mycenter}
\end{myframe}

\begin{myframe}
  \begin{center}
    {\hhuge bboxの値の単位は}\par
    \myvspace{1}
    {\hhuge \alert{ポイント(bp)}}\par
  \end{center}
\end{myframe}

\begin{myframe}
  \begin{mycenter}
    \begin{tikzpicture}
      \useasboundingbox[drawbb](-5.5,-4.5) rectangle (5.5,4.5);
      \begin{onlyenv}<1-3>
        \node[font=\LARGE] at (0,3.5)
          {bboxの値は\myiswhat};
      \end{onlyenv}
      \begin{onlyenv}<4>
        \node[font=\LARGE] at (0,3.5)
          {\mygood{普通の}\>bboxの値。};
      \end{onlyenv}
      \begin{onlyenv}<5>
        \node[font=\LARGE] at (0,3.5)
          {\mybad{バッドな}\>bboxの値。};
      \end{onlyenv}
      \node[right, text=structure, font=\bfseries] at (-4.8,1.5)
        {image.png};
      \node[imframe] at (-3,-0.5) (Image)
        {\pgfuseimage{texconf14png}};
      \node[right, textbox, font=\large] at (-1.5,-0.5)
        {\begin{minipage}{12em}
          \begin{itemize}
          \item PNG形式
          \item 解像度 = \alert{\bfseries 96\,dpi}
          \item 480\,px × 480\,px
          \end{itemize}
        \end{minipage}};
      \begin{onlyenv}<1>
        \node[text=black, font=\huge, xscale=0.8] at (0,-3.5) (Desc)
          {96\,px = 1\,in = 72\,bp};
        \draw[->, mygood, ultra thick]
          (2.4,-0.75)--(Desc.north);
      \end{onlyenv}
      \begin{onlyenv}<2>
        \node[text=black, font=\huge] at (0,-3.5) (Desc)
          {1\,px = 0.75\,bp};
      \end{onlyenv}
      \begin{onlyenv}<3>
        \node[text=black, font=\huge] at (0,-3.5) (Desc)
          {480\,px = 360\,bp};
        \draw[->, mygood, ultra thick]
          (0.4,-1.5)--(Desc.north);
      \end{onlyenv}
      \begin{onlyenv}<4>
        \node[text=black, font=\huge] at (0,-3.5) (Desc)
          {bb=[\textbf{0 0 \mygood{360 360}}]};
        \node[text=mygood, font=\huge] at (0.5,-2.4) {\facegood};
        \node[right, text=mygood, font=\Large] at (1,-2.4)
          {ポイント単位。};
      \end{onlyenv}
      \begin{onlyenv}<5>
        \node[text=black, font=\huge] at (0,-3.5) (Desc)
          {bb=[\textbf{0 0 \mybad{480 480}}]};
        \node[text=mybad, font=\huge] at (0.5,-2.4) {\facebad};
        \node[right, text=mybad, font=\Large] at (1,-2.4)
          {ピクセル単位。};
      \end{onlyenv}
    \end{tikzpicture}
  \end{mycenter}
\end{myframe}

\begin{myframe}
  \begin{mycenter}
    \begin{tikzpicture}
      \useasboundingbox[drawbb](-5.5,-4.5) rectangle (5.5,4.5);
      \node[text=black, font=\LARGE] at (0,3.5)
        {\mybad{“ピクセル単位”}\:してみよう\!};
      \node[textbox, font=\Large, opacity=0.8]
        at (0,0) (Desc)
        {\begin{minipage}{15em}
          \begin{itemize}
          \item \structure{さっきの例}の通りに\\
            PNG画像を\structure{実物大で}挿入。
            \par\smallskip
            \begin{tikzpicture}
              \node[badcode, font=\ttfamily\normalsize]
                at (0,0) (Part2)
                {\begin{minipage}{16em}
                  \myverb{\\includegraphics}\\
                  \myverb{\mbox{~~}[\alert{bb=0 0 480 480}}]%
                  \{image.png\}
                 \end{minipage}};
            \end{tikzpicture}
          \end{itemize}
        \end{minipage}};
    \end{tikzpicture}
  \end{mycenter}
\end{myframe}

\begin{mybgcolor}{mylbad}

\begin{myframe}
  \begin{mycenter}
    \begin{tikzpicture}
      \useasboundingbox[drawbb](-5.5,-4.5) rectangle (5.5,4.5);
      \node[font=\LARGE] at (0,4)
        {\alert{大昔の}{\TeX}環境では…};
      \node[font=\large] at (0,3)
        {(2005年頃の{\TeX}環境)};
      \node[imframe] at (0,-0.8) {\pgfuseimage{trial-lip-ok}};
      \node[text=mybad, font=\hhhuge] at (4.8,2.5) {\facebad};
    \end{tikzpicture}
  \end{mycenter}
\end{myframe}

\begin{myframe}
  \begin{mycenter}
    \begin{tikzpicture}
      \useasboundingbox[drawbb](-5.5,-4.5) rectangle (5.5,4.5);
      \node[font=\LARGE] at (0,4)
        {\alert{新しい}{\TeX}環境では…};
      \node[font=\large] at (0,3)
        {({\TeX} Live 2014)};
      \node[imframe] at (0,-0.8) {\pgfuseimage{trial-lip-ok}};
      \node[text=mybad, font=\hhhuge] at (4.8,2.5) {\facebad};
    \end{tikzpicture}
  \end{mycenter}
\end{myframe}

\end{mybgcolor}

\begin{mybgcolor}{mylbanned}

\begin{myframe}
  \begin{mycenter}
    \begin{tikzpicture}
      \useasboundingbox[drawbb](-5.5,-4.5) rectangle (5.5,4.5);
      \node[font=\LARGE] at (0,4)
        {\alert{少し古い}{\TeX}環境では…};
      \node[font=\large] at (0,3)
        {({\TeX} Live 2012)};
      \node[imframe] at (0,-0.8) {\pgfuseimage{trial-lip-ng}};
      \node[text=mybanned, font=\hhhuge] at (4.8,2.5) {\facebanned};
    \end{tikzpicture}
  \end{mycenter}
\end{myframe}

\begin{myframe}
  \begin{center}\color{mybanned}
    {\hhhuge 楽しい…{}?\par}
    \myvspace{1}
    {\Large \mytanoshy\par}
  \end{center}
\end{myframe}

\end{mybgcolor}

\begin{mybgcolor}{mydbanned}

\begin{myframe}
  \begin{center}\color{white}
    {\hhhhuge\bfseries 楽しくない\!\!\par}
    \myvspace{1}
    {\gigantic \facebanned\par}
  \end{center}
\end{myframe}

\end{mybgcolor}

\begin{mybgcolor}{mylgood}

\begin{myframe}
  \begin{mycenter}
    \begin{tikzpicture}
      \useasboundingbox[drawbb](-5.5,-4.5) rectangle (5.5,4.5);
      \node[text=black, font=\hhuge\bfseries, xscale=0.95]
        at (0,3) {\mygood{正しい}コードを書こう!};
      \node[text=mygood, font=\gigantic] at (0,0) {\facegood};
      \node[text=mygood, font=\fntVMin\LARGE, rotate=-90]
        at (-1.9,-2.4) {まさに};
      \node[text=mygood, font=\fntVMin\hhuge, rotate=-90]
        at (-2.7,-3.0) {正論。};
    \end{tikzpicture}
  \end{mycenter}
\end{myframe}

\end{mybgcolor}

% EOF
