%
% 32fbb.tex
% 「その② バウンディングボックス詐称」

%% 画像オブジェクト
\pgfdeclareimage[width=0.75\paperwidth]%
  {trial-fbb-ok}{image/trial-fbb-ok.pdf}
\pgfdeclareimage[width=0.75\paperwidth]%
  {trial-fbb-ng}{image/trial-fbb-ng.pdf}
%
\begin{mybgcolor}{mydbad}

\begin{myframe}
  \begin{center}\LARGE\color{white}
    \mybadbkname[0.5]{②}{バウンディング\\ボックス詐称}%
        {false bounding box}
  \end{center}
\end{myframe}

\end{mybgcolor}

\begin{myframe}
  \begin{mycenter}
    \begin{tikzpicture}
      \useasboundingbox[drawbb](-5.5,-4.5) rectangle (5.5,4.5);
      \node[font=\LARGE] at (0,3.5)
        {\begin{minipage}{12\zw}\centering
          バウンディングボックス\\
          (bbox)って何
        \end{minipage}};
      \begin{onlyenv}<1>
        \node[font=\Large] at (2.9,2.9)
          {\myiswhat};
      \end{onlyenv}
      \begin{onlyenv}<2>
        \node[font=\hhhhuge, text=mybanned] at (0,0)
          {説明省略。};
        \node[font=\huge\fntKiloji, text=structure] at (0,-3)
          {そう、アレです。};
      \end{onlyenv}
    \end{tikzpicture}
  \end{mycenter}
\end{myframe}

\begin{myframe}
  \begin{mycenter}
    \begin{tikzpicture}
      \useasboundingbox[drawbb](-5.5,-4.5) rectangle (5.5,4.5);
      \begin{onlyenv}<1-3>
        \node[font=\LARGE] at (0,3.5)
          {\begin{minipage}{10\zw}\centering
            各々の画像は\alert{固有の}\\
            bboxの値を持つ。
          \end{minipage}};
      \end{onlyenv}
      \begin{onlyenv}<4>
        \node[font=\LARGE] at (0,3.5)
          {\mygood{普通の}bbox指定の方法。};
      \end{onlyenv}
      \begin{onlyenv}<5>
        \node[font=\LARGE] at (0,3.5)
          {画像を\structure{“一部だけ”}挿入したい\!};
      \end{onlyenv}
      \begin{onlyenv}<6>
        \node[font=\LARGE] at (0,3.5)
          {\mybad{バッドな}bbox指定の方法。};
      \end{onlyenv}
      \begin{onlyenv}<1-2>
        \node[right, text=structure, font=\bfseries] at (-4.8,1.5)
          {image.eps};
        \node[imframe] at (-3,-0.5) (Image)
          {\pgfuseimage{texconf14eps}};
      \end{onlyenv}
      \begin{onlyenv}<2>
        \node[right, text=black, font=\large] at (-1.5,-0.5)
        {\begin{minipage}{12em}
          \begin{itemize}
          \item EPS形式
          \item bbox =\par
              \alert{[540 315 900 675]}
          \end{itemize}
        \end{minipage}};
      \end{onlyenv}
      \begin{onlyenv}<3-6>
        \node[right, text=structure, font=\bfseries] at (-4.8,1.5)
          {image.png};
        \node[imframe] at (-3,-0.5)
          {\pgfuseimage{texconf14png}};
        \node[right, text=black, font=\large] at (-1.5,-0.5)
        {\begin{minipage}{12em}
          \begin{itemize}
          \item PNG形式
          \item bbox =\par
              \alert{[0 0 360 360]}
          \end{itemize}
        \end{minipage}};
      \end{onlyenv}
      \begin{onlyenv}<5-6>
        \coordinate (I1) at (Image.south west);
        \coordinate (I2) at (Image.north east);
        \path let
          \p1=($(I1)!0.05!(I2)$),
          \p2=($(I1)!0.45!(I2)$),
          \p3=($(I1)!0.95!(I2)$)
        in 
          coordinate (P1) at (\x1,\y2)
          coordinate (P2) at (\x3,\y3);
        \begin{scope}[every node/.style={
          text=structure, font=\small, fill=white, opacity=0.75}]
          \begin{onlyenv}<5>
            \node[below] at (I1) {(0,0)};
            \node[above] at (I2) {(360,360)};
            \node[below] at ($(P1) + (-0.4,0)$) {(18,162)};
            \node[right] at ($(P2) + (0,-0.2)$) {(342,342)};
            \fill[structure] (I1) circle[radius=4pt];
            \fill[structure] (I2) circle[radius=4pt];
            \fill[structure] (P1) circle[radius=4pt];
            \fill[structure] (P2) circle[radius=4pt];
          \end{onlyenv}
          \draw[structure, very thick, dashed]
            (P1) rectangle (P2);
        \end{scope}
        \node[right, draw=structure, thick, text=black, font=\large]
          (T2) at (0.5,2)
        {\begin{minipage}{9em}
          \structure{“一部”} =\\
          \mybad{[18 162 342 342]}
        \end{minipage}};
        \draw[->, structure, thick]
          (T2.west) to[out=180, in=20] ($(P2)+(2pt,-1.0)$);
      \end{onlyenv}
      \begin{onlyenv}<3>
        \node[text=black, font=\Large] at (0, -3.5)
          {ビットマップ画像も同様。};
      \end{onlyenv}
      \begin{onlyenv}<4>
        \node[goodcode, font=\large, xscale=0.85]
          at (0,-3.5) (Src)
          {\myverb{\\includegraphics[\mygood{bb=0 0 360 360}]%
            \{\structure{image.png}\}}};
        \draw[->, mygood, very thick]
          (1.85,-1.4)--($(Src.north)+(1.20,-0.25)$);
        \node[text=mygood, font=\hhuge] at (2.5,-2.3) {\facegood};
        \node[text=mygood, font=\Large] at (4.0,-2.3) {普通。};
      \end{onlyenv}
      \begin{onlyenv}<6>
        \node[badcode, font=\large, xscale=0.85]
          at (0,-3.5) (Src)
          {\myverb{\\includegraphics[\mybad{bb=18 162 342 342}]%
            \{\structure{image.png}\}}};
        \draw[<->, mybad, very thick]
          (1.85,-1.4)--($(Src.north)+(1.25,-0.25)$);
        \draw[->, mybad, very thick]
          ($(T2.south)+(-1.5,0.25)$)--($(Src.north)+(-0.1,-0.25)$);
        \node[text=mybad, font=\hhuge] at (2.5,-2.3) {\facebad};
        \node[text=mybad, font=\Large] at (4.6,-2.3) {食い違う\!};
      \end{onlyenv}
    \end{tikzpicture}
  \end{mycenter}
\end{myframe}

\begin{myframe}
  \begin{mycenter}
    \begin{tikzpicture}
      \useasboundingbox[drawbb](-5.5,-4.5) rectangle (5.5,4.5);
      \node[font=\LARGE] at (0,4)
        {\mybad{“bbox詐称”}してみよう\!};
      \node[descbox, font=\Large, opacity=0.8]
        at (0,-0.6) (Desc)
        {\begin{minipage}{15em}
          \begin{itemize}
          \item ドライバは\mygood{当然}%
            “\myverb{dvipdfmx}”。
          \item \structure{さっきの例}の通りに\\
            PNG画像を\alert{“一部だけ”}挿入。
            \par\smallskip
            \begin{tikzpicture}
              \node[badcode, font=\ttfamily\normalsize]
                at (0,0) (Part2)
                {\begin{minipage}{16em}
                  \myverb{\\includegraphics}\\
                  \myverb{\mbox{~~}[\alert{bb=18 162 342 342}}]%
                  \{image.png\}
                 \end{minipage}};
            \end{tikzpicture}
          \end{itemize}
        \end{minipage}};
    \end{tikzpicture}
  \end{mycenter}
\end{myframe}

\begin{mybgcolor}{mylbad}

\begin{myframe}
  \begin{mycenter}
    \begin{tikzpicture}
      \useasboundingbox[drawbb](-5.5,-4.5) rectangle (5.5,4.5);
      \node[font=\LARGE] at (0,4)
        {\alert{古い}{\TeX}環境で処理すると…};
      \node[font=\large] at (0,3)
        {({\TeX} Live 2012)};
      \node[imframe] at (0,0) {\pgfuseimage{trial-fbb-ok}};
      \node[text=mybad, font=\hhhuge] at (-1.5,-3.5) {\facebad};
      \node[text=mybad, font=\hhuge] at (1.5,-3.5) {正常。};
    \end{tikzpicture}
  \end{mycenter}
\end{myframe}

\end{mybgcolor}

\begin{mybgcolor}{mylbanned}

\begin{myframe}
  \begin{mycenter}
    \begin{tikzpicture}
      \useasboundingbox[drawbb](-5.5,-4.5) rectangle (5.5,4.5);
      \node[font=\LARGE] at (0,4)
        {\alert{新しい}{\TeX}環境で処理すると…};
      \node[font=\large] at (0,3)
        {({\TeX} Live 2014)};
      \node[imframe] at (0,0) {\pgfuseimage{trial-fbb-ng}};
      \node[text=mybanned, font=\hhhuge] at (-1.5,-3.5) {\facebanned};
      \node[text=mybanned, font=\hhuge] at (1.5,-3.5) {破滅。};
    \end{tikzpicture}
  \end{mycenter}
\end{myframe}

\begin{myframe}
  \begin{center}\color{mybanned}
    {\hhhuge 楽しい\!\!\!\par}
    \myvspace{1}
    {\Large \mytanoshy\par}
  \end{center}
\end{myframe}

\end{mybgcolor}

% EOF
