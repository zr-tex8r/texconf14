%
% 02setting.tex
%   LaTeX レベルでの設定
%

%% フォント設定
\setjafontscale{0.925}
\setmainfont{Nishiki-teki}% にしき的フォント
\setsansfont[BoldFont=mplus-1p-heavy.ttf]%
            {mplus-1p-medium.ttf}% M+ 1P
\setmonofont{Inconsolata}
\setjamainfont{Nishiki-teki}% にしき的フォント
\setjasansfont[BoldFont=mplus-1p-heavy.ttf]%
              {mplus-1p-medium.ttf}% M+ 1P
\setjamonofont{mplus-1mn-regular.ttf}% M+ 1MN
\renewcommand\CJKfamilydefault{\CJKsfdefault}
\CJKsetecglue{\hspace{0.125em plus 0.05em minus 0.025em}}%
\newfontfamily\fntTermes{TeX Gyre Termes}
\newfontfamily\fntHeros{TeX Gyre Heros}
\newfontfamily\fntHerosCn{TeX Gyre Heros Cn}
\newfontfamily\fntPagella{TeX Gyre Pagella}
\setjafamilyfont{jPixel}{PixelMplus12-Regular.ttf}% PixelMplus
\newcommand*{\fntPixel}{\jafamily{jPixel}}
\setjafamilyfont{jKiloji}[BoldFont=kiloji - B]{kiloji}% きろ字
\newcommand*{\fntKiloji}{\jafamily{jKiloji}}
\setjafamilyfont{jMogaMin}[BoldFont=* Bold]{MogaExMincho}% MogaEx明朝
\newcommand*{\fntMin}{\jafamily{jMogaMin}\bfseries}% 太字
\setjafamilyfont{jMogaVMin}
  [BoldFont=* Bold, Vertical=RotatedGlyphs]{MogaExMincho}% MogaEx明朝
\newcommand*{\fntVMin}{\jafamily{jMogaVMin}\bfseries}% 太字
\newcommand*\fntTxtt{\usefont{T1}{txtt}{m}{n}}
\newfontfamily\fntDjvSans{DejaVu Sans}

%% Beamer設定
%\usetheme{Warsaw}
\setbeamertemplate{headline}[default]
\setbeamertemplate{navigation symbols}{} 
\usefonttheme{professionalfonts}
\setbeamercovered{transparent}
% 色
\definecolor{mycyan}{rgb}{0,1,1}% rgbのcyan
\definecolor{mygood}{rgb}{0,0.45,0.06}
\definecolor{mybad}{rgb}{0.6,0.1,0}
\definecolor{mybanned}{rgb}{0.4,0.4.0.4}
\colorlet{mygray}{black!55!white}
\colorlet{mylgood}{mygood!15!white}
\colorlet{mydbad}{mybad!75!black}
\colorlet{mylbad}{mybad!15!white}
\colorlet{mydbanned}{mybanned!75!black}
\colorlet{mylbanned}{mybanned!25!white}
{\usebeamercolor*{structure}
 \usebeamercolor*{alerted text}}
\colorlet{alert}{alerted text.fg}
\colorlet{structure}{structure.fg}
\newcommand*{\mygood}{\textcolor{mygood}}
\newcommand*{\mybad}{\textcolor{mybad}}
\newcommand*{\mybanned}{\textcolor{mybanned}}
\newenvironment{mybgcolor}[1]%
  {\setbeamercolor{background canvas}{bg=#1}}%
  {}

%% TikZはアレ
\tikzset{drawbb/.style={}}% \showcanvasframe で更新される
\tikzset{%
  imframe/.style={draw=black, fill=white, thin, inner sep=0.4pt},
  goodcode/.style={draw=mygood, very thick, fill=mylgood,
    text=black},
  badcode/.style={draw=mybad, very thick, fill=mylbad,
    text=black},
  examplecode/.style={draw=mydbanned, very thick, text=black},
  outbox/.style={very thick, fill=white, text=black},
  descbox/.style={draw=structure, thick, fill=white, text=black},
}

%% テキスト
% レイアウト
\newcommand*\mycompressitems{\setlength{\itemsep}{0pt}}
\newcommand*\myvspace[1]{\par\vspace{#1\baselineskip}}
\newcommand*\mynormvspace[1]{{\par\normalsize\vspace{#1\baselineskip}}}
\newenvironment{mycenter}
  {\begin{center}\myhac}%
  {\end{center}}
\newenvironment{myframe}
  {\begin{frame}[plain]\relax}%
  {\end{frame}}
% 装飾
\newcommand*{\myiswhat}{\makebox[0pt][l]{\sffamily\bfseries\alert{?}}}
\newcommand*\・{\makebox[0.5\zw]{・}}
\newcommand*\!{\makebox[0.5\zw]{!}}

% フェイスマーク
\newcommand*\facechar[1]{{\fntDjvSans\symbol{"#1}}}
\newcommand*\facegood{\facechar{263A}}% WHITE SMILING FACE
\newcommand*\facebad{\facechar{1F601}}% GRINNING FACE WITH SMILING EYES
\newcommand*\facebanned{\facechar{1F62D}}% LOUDLY CRYING FACE
\newcommand*\✌{{\fntDjvSans\raisebox{-0.1em}{\scalebox{1.8}[1.5]{%
  \makebox[0.5em]{✌}}}}}
\newcommand*\mytanoshy{\textrawen{%
\✌('ω'\✌ )三\✌('ω')\✌三( \✌'ω')\✌%
}}

%% その他の設定
\newenvironment{mygroup}{}{}
\newcommand*{\myhac}[1][-0.25]{\hspace*{#1cm}}

% EOF
