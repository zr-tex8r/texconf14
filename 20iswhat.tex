%
% 20badkh.tex
% 「バッド・ノウハウの魔」
%

%% 画像オブジェクト
\pgfdeclareimage[]%
  {machigatteru}{image/machigatteru.pdf}
\pgfdeclareimage[width=0.8\paperwidth]%
  {texlive2015}{image/image-texlive2015.png}

\begin{myframe}
  \begin{center}
    {\hhhuge \mybad{バッドノウハウ}}%
    \visible<1>{\huge\myiswhat}
    \begin{visibleenv}<2>
      \myvspace{0.5}
      \rotatebox{-90}{\hhhuge =}
      \myvspace{0.5}
      {\hhhhuge \fntPixel\myoutlined{black}{alert}{バグ技}}%
    \end{visibleenv}
  \end{center}
\end{myframe}

\begin{myframe}
  \begin{mycenter}
    \begin{tikzpicture}
      \useasboundingbox[drawbb](-5.5,-4.5) rectangle (5.5,4.5);
      \draw[structure, very thick](0,-4)--(0,3);
      \node[above, fill=mygood, text=white, font=\Large]
        at (-3,3) {\makebox[6\zw]{正攻法}};
      \node[above, fill=mybad, text=white, font=\Large, xscale=0.8]
        at (3,3) {バッド\・ノウハウ};
      \node[above,font=\rmfamily\LARGE]
        at (0,3) {vs};
      \begin{onlyenv}<2->
         \node[goodcode, font=\sffamily]
           at (-3,1) (GoodIn)
           {\begin{minipage}{5cm}
             \myverb{\\begin\{document\}}\par
             \medskip
             \hspace*{1em}{\large 正しいコード\par}
             \medskip
             \myverb{\\end\{document\}}\par
           \end{minipage}};
      \end{onlyenv}
      \begin{onlyenv}<3->
         \node[outbox, draw=mygood, font=\fntMin]
           at (-3,-1.5) (GoodOut)
           {\begin{minipage}{5cm}
             \large 正しい出力結果
           \end{minipage}};
         \draw[->, ultra thick, mygood]
           (GoodIn.south) -- (GoodOut.north);
      \end{onlyenv}
      \begin{onlyenv}<4->
         \node[text=mygood, font=\LARGE] at (-3,-3.5)
           {{\huge\facegood} 当然。};
      \end{onlyenv}
      \begin{onlyenv}<2>
        \node[font=\Large] at (-3,-3.5)
          {“仕様に従った”};
      \end{onlyenv}
      \begin{onlyenv}<5->
         \node[badcode, font=\sffamily]
           at (3,1) (BadIn)
           {\begin{minipage}{5cm}
             \myverb{\\begin\{document\}}\par
             \medskip
             \hspace*{1em}{\large\rmfamily
                 \scalebox{0.9}[1]{マチガッテルコード}\par}
             \medskip
             \myverb{\\end\{document\}}\par
           \end{minipage}};
      \end{onlyenv}
      \begin{onlyenv}<6->
         \node[outbox, draw=mybad, font=\fntMin]
           at (3,-1.5) (BadOut)
           {\begin{minipage}{5cm}
             \large 正しい出力結果
           \end{minipage}};
         \draw[->, ultra thick, mybad]
           (BadIn.south) -- (BadOut.north);
      \end{onlyenv}
      \begin{onlyenv}<7->
         \node[text=mybad, font=\LARGE] at (3,-3.5)
           {{\huge\facebad} 素敵。};
      \end{onlyenv}
      \begin{onlyenv}<5>
        \node[font=\Large] at (3,-3.5)
          {“仕様は\alert{未定義}”};
      \end{onlyenv}
    \end{tikzpicture}%
  \end{mycenter}
\end{myframe}

\begin{myframe}
  \begin{center}
    {\huge バージョンアップ!\par}
    \myvspace{0.5}
    \pgfuseimage{texlive2015}
  \end{center}
\end{myframe}

\begin{myframe}
  \begin{mycenter}
    \begin{tikzpicture}
      \useasboundingbox[drawbb](-5.5,-4.5) rectangle (5.5,4.5);
      \draw[structure, very thick](0,-4)--(0,3);
      \node[above, fill=mygood, text=white, font=\Large]
        at (-3,3) {\makebox[6\zw]{正攻法}};
      \node[above, fill=mybad, text=white, font=\Large, xscale=0.8]
        at (3,3) {バッド\・ノウハウ};
      \node[above,font=\rmfamily\LARGE]
        at (0,3) {vs};
      \begin{onlyenv}<1->
         \node[goodcode, font=\sffamily] 
           at (-3,1) (GoodIn)
           {\begin{minipage}{5cm}
             \myverb{\\begin\{document\}}\par
             \medskip
             \hspace*{1em}{\large 正しいコード\par}
             \medskip
             \myverb{\\end\{document\}}\par
           \end{minipage}};
      \end{onlyenv}
      \begin{onlyenv}<2->
         \node[outbox, draw=mygood, font=\fntMin]
           at (-3,-1.5) (GoodOut)
           {\begin{minipage}{5cm}
             \large 正しい出力結果
           \end{minipage}};
         \draw[->, ultra thick, mygood]
           (GoodIn.south) -- (GoodOut.north);
         \node[text=mygood, font=\LARGE] at (-3,-3.5)
           {{\huge\facegood} 当然。};
      \end{onlyenv}
      \begin{onlyenv}<1>
        \node[font=\Large] at (-3,-3.5)
          {“仕様に従った”};
      \end{onlyenv}
      \begin{onlyenv}<1->
         \node[badcode, font=\sffamily]
           at (3,1) (BadIn)
           {\begin{minipage}{5cm}
             \myverb{\\begin\{document\}}\par
             \medskip
             \hspace*{1em}{\large\rmfamily
                 \scalebox{0.9}[1]{マチガッテルコード}\par}
             \medskip
             \myverb{\\end\{document\}}\par
           \end{minipage}};
      \end{onlyenv}
      \begin{onlyenv}<3->
         \node[outbox, draw=mybad, inner sep=0pt]
           at (3,-1.5) (BadOut)
           {\pgfuseimage{machigatteru}};
         \draw[->, ultra thick, mybad]
           (BadIn.south) -- (BadOut.north);
      \end{onlyenv}
      \begin{onlyenv}<3->
         \node[text=mybanned, font=\LARGE] at (3,-3.5)
           {{\huge\facebanned} 破滅\!};
      \end{onlyenv}
      \begin{onlyenv}<1-2>
        \node[font=\Large] at (3,-3.5)
          {“仕様は\alert{未定義}”};
      \end{onlyenv}
    \end{tikzpicture}%
  \end{mycenter}
\end{myframe}

\begin{mybgcolor}{mydbanned}

\begin{myframe}
  \begin{center}\LARGE\color{white}
    \textcolor{mylbanned}{\bfseries 封印}された\\
    バッド\・ノウハウは\par
    \myvspace{0.5}
    \textcolor{mylbanned}{\hhhhhuge\rmfamily 破滅}\par
    \myvspace{0.5}
    を生む\par
    \myvspace{0.5}
    {\hhhuge \textcolor{mylbanned}{\facebanned}}\par
  \end{center}
\end{myframe}

\end{mybgcolor}

% EOF
