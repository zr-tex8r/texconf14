\documentclass[xetex,14pt]{beamer}
\usepackage{01setting}% TeXレベルの設定
\usepackage{manfnt}% 「Dangerous bend」マーク
\usepackage{zxjatype}
\usepackage{metalogo}
\usepackage{tikz}% TikZはアレ
%
% 02setting.tex
%   LaTeX レベルでの設定
%

%% フォント設定
\setjafontscale{0.925}
\setmainfont{Nishiki-teki}% にしき的フォント
\setsansfont[BoldFont=mplus-1p-heavy.ttf]%
            {mplus-1p-medium.ttf}% M+ 1P
\setmonofont{Inconsolata}
\setjamainfont{Nishiki-teki}% にしき的フォント
\setjasansfont[BoldFont=mplus-1p-heavy.ttf]%
              {mplus-1p-medium.ttf}% M+ 1P
\setjamonofont{mplus-1mn-regular.ttf}% M+ 1MN
\renewcommand\CJKfamilydefault{\CJKsfdefault}
\CJKsetecglue{\hspace{0.125em plus 0.05em minus 0.025em}}%
\newfontfamily\fntTermes{TeX Gyre Termes}
\newfontfamily\fntHeros{TeX Gyre Heros}
\newfontfamily\fntHerosCn{TeX Gyre Heros Cn}
\newfontfamily\fntPagella{TeX Gyre Pagella}
\newcommand*\fntTxtt{\usefont{T1}{txtt}{m}{n}}
%% Beamer設定
%\usetheme{Warsaw}
\setbeamertemplate{headline}[default]
\setbeamertemplate{navigation symbols}{} 
\usefonttheme{professionalfonts}
\setbeamercovered{transparent}
\definecolor{mycyan}{rgb}{0,1,1}% rgbのcyan
\definecolor{mygood}{rgb}{0,0.45,0.06}
\definecolor{mybad}{rgb}{0.45,0.05,0}
\definecolor{myban}{rgb}{0.4,0.4.0.4}
\colorlet{mygray}{black!55!white}
\colorlet{mylgood}{mygood!25!white}
\colorlet{mylbad}{mybad!25!white}
\colorlet{mylban}{myban!25!white}

% 基本色
{\usebeamercolor*{structure}
 \usebeamercolor*{alerted text}}
\colorlet{alert}{alerted text.fg}
\colorlet{structure}{structure.fg}
% 完全暗黒世界
\setbeamercolor{my dark world}{fg=black!15!white,bg=black}
{\usebeamercolor*{my dark world}}
\colorlet{my dark alert}{red!80!black}
%% 独自Beamer要素
\setbeamercolor{my good block}{bg=mylgood}
\setbeamercolor{my bad block}{bg=mylbad}
\setbeamercolor{my banned block}{bg=mylban}
%% テキストレイアウト
\newcommand*\mycompressitems{\setlength{\itemsep}{0pt}}
\newcommand\mypara[2]{\par{#1#2\par}}
\newcommand\latexmood[1]{\mbox{}{\color{mylatex}#1}}
\newcommand\texmood[1]{\mbox{}{\color{mytex}#1}}
%% その他の設定
\newenvironment{mygroup}{}{}
\newcommand*{\myhac}{\hspace*{-0.2cm}}
\newcommand*\myvspace[1]{\par\vspace{#1\baselineskip}}
\newcommand*\mynormvspace[1]{{\par\normalsize\vspace{#1\baselineskip}}}
% EOF
% LaTeXレベルの設定
%----------------------------------------------------------------------
%% 文書情報
\title{dvipdfmxと3つのバッド・ノウハウ}
\author[八登崇之]{八登崇之 (Takayuki YATO)}
\date{2014年11月8日 {\TeX}ユーザの集い 2014}
\subject{dvipdfmxはアレ}
\keywords{\TeX,\LaTeX}
%%
\begin{document}
%=======================================================================

%
% 10title.tex
% 「タイトル」
%
\begin{mygroup}
\setbeamercolor{background canvas}{bg=mybad}
\begin{frame}[plain]{}\color{white}%
  \begin{center}\myhac
    \begin{tikzpicture}
        [every node/.style={font=\rmfamily}]
      \node at(-1.5,5.0) {\color{mygood!40!white}\hhhhuge dvipdfmx};
      \node at (0,3.7) {\color{white}\Large と};
      \node at (0,2.5) {\color{white}{\hhhuge 3}{\Large つの}};
      \node at (0,1.3) {\color{white}\Large の};
      \node at (1,0) {\color{mybad!25!white}\hhhuge バッド・ノウハウ};
    \end{tikzpicture}
    \myvspace{1}
    {{\large 八登\ 崇之}\quad(Takayuki YATO)}\par
    \smallskip
    {\small{\TeX}ユーザの集い2014\qquad 2014年11月8日}
    aa
  \end{center}
\end{frame}

\begin{frame}[plain]{}\color{white}\large%
  \begin{center}\begin{minipage}{15em}
  この物語{\small (LT)}は
  \par\myvspace{0.75}\hspace*{1em}
  \textcolor{mylbad}{dvipdfmx}で
  \par\myvspace{0.75}\hspace*{2em}
  \textcolor{mylbad}{画像を挿入}するときの
  \par\myvspace{0.75}\hspace*{3em}
  {\LARGE\bfseries\textcolor{myban!50!white}{“封印”}}された
  \par\myvspace{0.75}\hspace*{4em}
  {\bfseries\textcolor{mylbad}{バッドノウハウ}}を
  \par\myvspace{0.75}\hspace*{5em}
  記したものである。
  \end{minipage}\end{center}
\end{frame}

\end{mygroup}

% EOF
% タイトル
\section{バッド・ノウハウの魔}
%
% 20badkh.tex
% 「バッド・ノウハウの魔」
%
% EOF

\section{その① ドライバ詐称}
%
% 30fdn.tex
% 「その① ドライバ偽装」
%
% EOF

\section{その② バウンディングボックス詐称}
%
% 31fbb.tex
% 「その② バウンディングボックス詐称」
%
% EOF

\section{その③ 面倒なのでピクセル単位}
%
% 32lip.tex
% 「その③ 面倒なのでピクセル単位」
%
% EOF

\section{正しさのすすめ}
%
% 40justice.tex
% 「正しさのすすめ」
%
% EOF


\end{document}
% EOF
\iffalse%

◇
dvipdfmxと
3つのバッド・ノウハウ

◇
この物語《LT》は
LaTeX + dvipdfmxでの
画像の挿入に関する
封印された
バッドノウハウの
記録である。

◇
バッド・ノウハウ
is
何

◇
このLTにおいて!
バッド・ノウハウ
is
バグ技

◇
正攻法       バッド・ノウハウ
正しいコード    /間違ったコード/
↓              
欲しい出力      (仕様上結果は未定義)
;-)
〈つまらない〉

is
バグ技

◇
正攻法       バッド・ノウハウ
正しいコード    /間違ったコード/
↓              ↓偶然
欲しい出力      欲しい出力
;-)             8-P イイネ!

バージョンアップ!

◇
正攻法       バッド・ノウハウ
正しいコード    /間違ったコード/
↓              ↓偶然
欲しい出力      .間'違:っ:た 出  力
;-)             X-O ウワァァァァ

◇
バッドノウハウ
その①
ドライバ詐称
false driver name

◇
\usepackage[dvipdfmx]{graphicx}
;-)

[zr@texconf14]$ dvipdfmx talk.dvi


◇
\usepackage[dvipdfm]{graphicx}
8-P

[zr@texconf14]$ dvipdfmx talk.dvi

◇
\usepackage[dvips]{graphicx}
\usepackage[dviout]{graphicx}


◇
(ソース①)

◇
TeX Live 2012
8-P

◇
TeX Live 2013
X-O

◇
楽しい!!!

◇
バッドノウハウ
その②
バウンディングボックス詐称
false bounding box

◇
画像ファイルは固有の
“バウンディングボックス”(bbox)
を持っている。

texconf.pdf

bboxの値
[100 100 500 500]

◇
画像ファイルは固有の
“バウンディングボックス”(bbox)
を持っている。

texconf.png

bboxの値
[0 0 400 400]


◇
(extractbb しない場合)
正しいbboxの値の指定が必要。

\includegraphics[
  bb=0 0 400 400,
  width=6cm]{texconf.png}
:-)

◇
間違ったbboxの値の指定。
(8-P「画像の一部を切り出したい」)

\includegraphics[
  bb=200 300 300 400,
  width=6cm]{texconf.png}
8-P

◇
TeX Live 2012
8-P

◇
TeX Live 最新
X-O

◇
楽しい!!!

◇
バッドノウハウ
その③
面倒なのでピクセル単位
lazily-in-pixels

◇
texconf.png
96dpi
400x400

正しいbboxの値は?


◇
texconf.png
96dpi
400x400

解像度は 96dpi
∴ 96 px = 1 in = 72 bp
∴ 400 px = 300bp

◇
texconf.png
96dpi
400x400

bb=[0 0 300 300]
:-)

◇
texconf.png
96dpi
400x400

bb=[0 0 400 400]
8-P

◇
2006年頃
8-P

◇
最新
X-O

◇
楽しくない!
X-O

◇
正しいコードを書こう!
;-)
まさに正論

◇
dvipdfm


◇
dvipdfm
は 2013年7月を以て
終了
致しました。
:-) Thank you & Good bye! :-)

◇
ステップ0
dvipdfm ×
dvipdfmx
を使う。

◇
ステップ①
(現状の TeX Live では)
extractbb の自動起動を有効にする。
その方法は…

◇
TeX Wiki 見ろ。
以上。

◇
ステップ③
dvipdfmx ドライバを指定する

\usepackage[dvipdfmx]{graphicx}
\usepackage[dvipdfmx]{color}

◇
ステップ④
\includegraphics する。
bb は付けない。

\includegraphics[width=6cm]{texconf.png}

◇
ステップ⑤
結果。

〈なんと画像がちゃんと入ります。すごいですね。〉

◇
楽しい

◇
Happy
   TeXing!
〈以上です〉

\fi%
