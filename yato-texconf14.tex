\documentclass[xetex,14pt]{beamer}
\usepackage{01setting}% TeXレベルの設定
%\usepackage{manfnt}% 「Dangerous bend」マーク
\usepackage{zxjatype}
%\usepackage{metalogo}
\usepackage{comment}
\usepackage{tikz}% TikZはアレ
\usetikzlibrary{calc}
%
% 02setting.tex
%   LaTeX レベルでの設定
%

%% フォント設定
\setjafontscale{0.925}
\setmainfont{Nishiki-teki}% にしき的フォント
\setsansfont[BoldFont=mplus-1p-heavy.ttf]%
            {mplus-1p-medium.ttf}% M+ 1P
\setmonofont{Inconsolata}
\setjamainfont{Nishiki-teki}% にしき的フォント
\setjasansfont[BoldFont=mplus-1p-heavy.ttf]%
              {mplus-1p-medium.ttf}% M+ 1P
\setjamonofont{mplus-1mn-regular.ttf}% M+ 1MN
\renewcommand\CJKfamilydefault{\CJKsfdefault}
\CJKsetecglue{\hspace{0.125em plus 0.05em minus 0.025em}}%
\newfontfamily\fntTermes{TeX Gyre Termes}
\newfontfamily\fntHeros{TeX Gyre Heros}
\newfontfamily\fntHerosCn{TeX Gyre Heros Cn}
\newfontfamily\fntPagella{TeX Gyre Pagella}
\newcommand*\fntTxtt{\usefont{T1}{txtt}{m}{n}}
%% Beamer設定
%\usetheme{Warsaw}
\setbeamertemplate{headline}[default]
\setbeamertemplate{navigation symbols}{} 
\usefonttheme{professionalfonts}
\setbeamercovered{transparent}
\definecolor{mycyan}{rgb}{0,1,1}% rgbのcyan
\definecolor{mygood}{rgb}{0,0.45,0.06}
\definecolor{mybad}{rgb}{0.45,0.05,0}
\definecolor{myban}{rgb}{0.4,0.4.0.4}
\colorlet{mygray}{black!55!white}
\colorlet{mylgood}{mygood!25!white}
\colorlet{mylbad}{mybad!25!white}
\colorlet{mylban}{myban!25!white}

% 基本色
{\usebeamercolor*{structure}
 \usebeamercolor*{alerted text}}
\colorlet{alert}{alerted text.fg}
\colorlet{structure}{structure.fg}
% 完全暗黒世界
\setbeamercolor{my dark world}{fg=black!15!white,bg=black}
{\usebeamercolor*{my dark world}}
\colorlet{my dark alert}{red!80!black}
%% 独自Beamer要素
\setbeamercolor{my good block}{bg=mylgood}
\setbeamercolor{my bad block}{bg=mylbad}
\setbeamercolor{my banned block}{bg=mylban}
%% テキストレイアウト
\newcommand*\mycompressitems{\setlength{\itemsep}{0pt}}
\newcommand\mypara[2]{\par{#1#2\par}}
\newcommand\latexmood[1]{\mbox{}{\color{mylatex}#1}}
\newcommand\texmood[1]{\mbox{}{\color{mytex}#1}}
%% その他の設定
\newenvironment{mygroup}{}{}
\newcommand*{\myhac}{\hspace*{-0.2cm}}
\newcommand*\myvspace[1]{\par\vspace{#1\baselineskip}}
\newcommand*\mynormvspace[1]{{\par\normalsize\vspace{#1\baselineskip}}}
% EOF
% LaTeXレベルの設定
%\showcanvasframe
%----------------------------------------------------------------------
%% 文書情報
\title{dvipdfmxと3つのバッド・ノウハウ}
\author[八登崇之]{八登崇之 (Takayuki YATO)}
\date{2014年11月8日 {\TeX}ユーザの集い 2014}
\subject{dvipdfmxはアレ}
\keywords{\TeX,\LaTeX}
%%
\begin{document}
%=======================================================================

%
% 10title.tex
% 「タイトル」
%
\begin{mygroup}
\setbeamercolor{background canvas}{bg=mybad}
\begin{frame}[plain]{}\color{white}%
  \begin{center}\myhac
    \begin{tikzpicture}
        [every node/.style={font=\rmfamily}]
      \node at(-1.5,5.0) {\color{mygood!40!white}\hhhhuge dvipdfmx};
      \node at (0,3.7) {\color{white}\Large と};
      \node at (0,2.5) {\color{white}{\hhhuge 3}{\Large つの}};
      \node at (0,1.3) {\color{white}\Large の};
      \node at (1,0) {\color{mybad!25!white}\hhhuge バッド・ノウハウ};
    \end{tikzpicture}
    \myvspace{1}
    {{\large 八登\ 崇之}\quad(Takayuki YATO)}\par
    \smallskip
    {\small{\TeX}ユーザの集い2014\qquad 2014年11月8日}
    aa
  \end{center}
\end{frame}

\begin{frame}[plain]{}\color{white}\large%
  \begin{center}\begin{minipage}{15em}
  この物語{\small (LT)}は
  \par\myvspace{0.75}\hspace*{1em}
  \textcolor{mylbad}{dvipdfmx}で
  \par\myvspace{0.75}\hspace*{2em}
  \textcolor{mylbad}{画像を挿入}するときの
  \par\myvspace{0.75}\hspace*{3em}
  {\LARGE\bfseries\textcolor{myban!50!white}{“封印”}}された
  \par\myvspace{0.75}\hspace*{4em}
  {\bfseries\textcolor{mylbad}{バッドノウハウ}}を
  \par\myvspace{0.75}\hspace*{5em}
  記したものである。
  \end{minipage}\end{center}
\end{frame}

\end{mygroup}

% EOF
% タイトル
\section{バッド・ノウハウ is 何}
%
% 20badkh.tex
% 「バッド・ノウハウの魔」
%

%% 画像オブジェクト
\pgfdeclareimage[]%
  {machigatteru}{image/machigatteru.pdf}
\pgfdeclareimage[width=0.8\paperwidth]%
  {texlive2015}{image/image-texlive2015.png}

\begin{myframe}
  \begin{center}
    {\hhhuge \mybad{バッドノウハウ}}%
    \visible<1>{\huge\myiswhat}
    \begin{visibleenv}<2>
      \myvspace{0.5}
      \rotatebox{-90}{\hhhuge =}
      \myvspace{0.5}
      {\hhhhuge \fntPixel\myoutlined{black}{alert}{バグ技}}%
    \end{visibleenv}
  \end{center}
\end{myframe}

\begin{myframe}
  \begin{mycenter}
    \begin{tikzpicture}
      \useasboundingbox[drawbb](-5.5,-4.5) rectangle (5.5,4.5);
      \draw[structure, very thick](0,-4)--(0,3);
      \node[above, fill=mygood, text=white, font=\Large]
        at (-3,3) {\makebox[6\zw]{正攻法}};
      \node[above, fill=mybad, text=white, font=\Large, xscale=0.8]
        at (3,3) {バッド\・ノウハウ};
      \node[above,font=\rmfamily\LARGE]
        at (0,3) {vs};
      \begin{onlyenv}<2->
         \node[goodcode, font=\sffamily]
           at (-3,1) (GoodIn)
           {\begin{minipage}{5cm}
             \myverb{\\begin\{document\}}\par
             \medskip
             \hspace*{1em}{\large 正しいコード\par}
             \medskip
             \myverb{\\end\{document\}}\par
           \end{minipage}};
      \end{onlyenv}
      \begin{onlyenv}<3->
         \node[outbox, draw=mygood, font=\fntMin]
           at (-3,-1.5) (GoodOut)
           {\begin{minipage}{5cm}
             \large 正しい出力結果
           \end{minipage}};
         \draw[->, ultra thick, mygood]
           (GoodIn.south) -- (GoodOut.north);
      \end{onlyenv}
      \begin{onlyenv}<4->
         \node[text=mygood, font=\LARGE] at (-3,-3.5)
           {{\huge\facegood} 当然。};
      \end{onlyenv}
      \begin{onlyenv}<2>
        \node[font=\Large] at (-3,-3.5)
          {“仕様に従った”};
      \end{onlyenv}
      \begin{onlyenv}<5->
         \node[badcode, font=\sffamily]
           at (3,1) (BadIn)
           {\begin{minipage}{5cm}
             \myverb{\\begin\{document\}}\par
             \medskip
             \hspace*{1em}{\large\rmfamily
                 \scalebox{0.9}[1]{マチガッテルコード}\par}
             \medskip
             \myverb{\\end\{document\}}\par
           \end{minipage}};
      \end{onlyenv}
      \begin{onlyenv}<6->
         \node[outbox, draw=mybad, font=\fntMin]
           at (3,-1.5) (BadOut)
           {\begin{minipage}{5cm}
             \large 正しい出力結果
           \end{minipage}};
         \draw[->, ultra thick, mybad]
           (BadIn.south) -- (BadOut.north);
      \end{onlyenv}
      \begin{onlyenv}<7->
         \node[text=mybad, font=\LARGE] at (3,-3.5)
           {{\huge\facebad} 素敵。};
      \end{onlyenv}
      \begin{onlyenv}<5>
        \node[font=\Large] at (3,-3.5)
          {“仕様は\alert{未定義}”};
      \end{onlyenv}
    \end{tikzpicture}%
  \end{mycenter}
\end{myframe}

\begin{myframe}
  \begin{center}
    {\huge バージョンアップ!\par}
    \myvspace{0.5}
    \pgfuseimage{texlive2015}
  \end{center}
\end{myframe}

\begin{myframe}
  \begin{mycenter}
    \begin{tikzpicture}
      \useasboundingbox[drawbb](-5.5,-4.5) rectangle (5.5,4.5);
      \draw[structure, very thick](0,-4)--(0,3);
      \node[above, fill=mygood, text=white, font=\Large]
        at (-3,3) {\makebox[6\zw]{正攻法}};
      \node[above, fill=mybad, text=white, font=\Large, xscale=0.8]
        at (3,3) {バッド\・ノウハウ};
      \node[above,font=\rmfamily\LARGE]
        at (0,3) {vs};
      \begin{onlyenv}<1->
         \node[goodcode, font=\sffamily] 
           at (-3,1) (GoodIn)
           {\begin{minipage}{5cm}
             \myverb{\\begin\{document\}}\par
             \medskip
             \hspace*{1em}{\large 正しいコード\par}
             \medskip
             \myverb{\\end\{document\}}\par
           \end{minipage}};
      \end{onlyenv}
      \begin{onlyenv}<2->
         \node[outbox, draw=mygood, font=\fntMin]
           at (-3,-1.5) (GoodOut)
           {\begin{minipage}{5cm}
             \large 正しい出力結果
           \end{minipage}};
         \draw[->, ultra thick, mygood]
           (GoodIn.south) -- (GoodOut.north);
         \node[text=mygood, font=\LARGE] at (-3,-3.5)
           {{\huge\facegood} 当然。};
      \end{onlyenv}
      \begin{onlyenv}<1>
        \node[font=\Large] at (-3,-3.5)
          {“仕様に従った”};
      \end{onlyenv}
      \begin{onlyenv}<1->
         \node[badcode, font=\sffamily]
           at (3,1) (BadIn)
           {\begin{minipage}{5cm}
             \myverb{\\begin\{document\}}\par
             \medskip
             \hspace*{1em}{\large\rmfamily
                 \scalebox{0.9}[1]{マチガッテルコード}\par}
             \medskip
             \myverb{\\end\{document\}}\par
           \end{minipage}};
      \end{onlyenv}
      \begin{onlyenv}<3->
         \node[outbox, draw=mybad, inner sep=0pt]
           at (3,-1.5) (BadOut)
           {\pgfuseimage{machigatteru}};
         \draw[->, ultra thick, mybad]
           (BadIn.south) -- (BadOut.north);
      \end{onlyenv}
      \begin{onlyenv}<3->
         \node[text=mybanned, font=\LARGE] at (3,-3.5)
           {{\huge\facebanned} 破滅\!};
      \end{onlyenv}
      \begin{onlyenv}<1-2>
        \node[font=\Large] at (3,-3.5)
          {“仕様は\alert{未定義}”};
      \end{onlyenv}
    \end{tikzpicture}%
  \end{mycenter}
\end{myframe}

\begin{mybgcolor}{mydbanned}

\begin{myframe}
  \begin{center}\LARGE\color{white}
    \textcolor{mylbanned}{\bfseries 封印}された\\
    バッド\・ノウハウは\par
    \myvspace{0.5}
    \textcolor{mylbanned}{\hhhhhuge\rmfamily 破滅}\par
    \myvspace{0.5}
    を生む\par
    \myvspace{0.5}
    {\hhhuge \textcolor{mylbanned}{\facebanned}}\par
  \end{center}
\end{myframe}

\end{mybgcolor}

% EOF

\section{バッド・ノウハウ三銃士を連れてきたよ}
%
% 30example.tex
% 「バッド・ノウハウ三銃士を連れてきたよ」
%

%% 画像オブジェクト
\pgfdeclareimage[width=0.25\paperwidth]%
  {texconf14eps}{image/texconf14logo-eps.eps}
\pgfdeclareimage[width=0.25\paperwidth]%
  {texconf14pdf}{image/texconf14logo-pdf.pdf}
\pgfdeclareimage[width=0.25\paperwidth]%
  {texconf14png}{image/texconf14logo-png.png}


%% 名前紹介のレイアウト
\newcommand*{\mybadbkname}[4][1]{%
  {\large\rmfamily バッド\・ノウハウ\quad その%
    {\LARGE\rmfamily #2}%
  \par}
  \myvspace{#1}
  {\hhhuge\bfseries #3\par}%
  \myvspace{#1}
  {\LARGE\rmfamily #4}%
}

\begin{mybgcolor}{mydbanned}

\newcommand*{\mybkno}[1]{{\rmfamily\LARGE#1}\quad}
\begin{myframe}
  \begin{center}\color{white}
    {\LARGE
      \textcolor{mylbanned}{\bfseries 封印}された\\
      \textcolor{mylbad}{バッド\・ノウハウ}たち。\par}
    \myvspace{2}
    {\Large\begin{minipage}{15em}
      \mybkno{①}ドライバ詐称
      \par\myvspace{1}
      \mybkno{②}バウンディングボックス詐称
      \par\myvspace{1}
      \mybkno{③}面倒だからピクセル単位
    \end{minipage}}
  \end{center}
\end{myframe}

\end{mybgcolor}

% EOF

\subsection{その① ドライバ詐称}
%
% 31fdn.tex
% 「その① ドライバ詐称」
%

%% 画像オブジェクト
\pgfdeclareimage[width=0.5\paperwidth]%
  {trial-fdn-ok}{image/trial-fdn-ok.pdf}
\pgfdeclareimage[width=0.5\paperwidth]%
  {trial-fdn-ng}{image/trial-fdn-ng.pdf}
%% 色
\colorlet{myprompt}{green!75!black}

\begin{mybgcolor}{mydbad}

\begin{myframe}
  \begin{center}\LARGE\color{white}
    \mybadbkname{①}{ドライバ詐称}{false driver name}
  \end{center}
\end{myframe}

\end{mybgcolor}

\begin{myframe}
  \begin{mycenter}
    \begin{tikzpicture}
      \useasboundingbox[drawbb](-5.5,-3) rectangle (5.5,5);
      \begin{onlyenv}<1-2>
        \node[font=\LARGE] at (0,4)
          {ドライバ指定};
      \end{onlyenv}
      \begin{onlyenv}<1>
        \node[font=\Large] at (2.6,3.9)
          {\myiswhat};
      \end{onlyenv}
      \begin{onlyenv}<3>
        \node[font=\LARGE] at (0,4)
          {\mygood{普通の}ドライバ指定};
        \node[text=mygood, font=\hhuge] at (-2.7,-0.5) {\facegood};
      \end{onlyenv}
      \begin{onlyenv}<4-6>
        \node[font=\LARGE] at (0,4)
          {\mybad{バッドな}ドライバ指定};
        \node[text=mybad, font=\hhuge] at (-2.7,-0.5) {\facebad};
      \end{onlyenv}
      \begin{onlyenv}<3-6>
        \node[left, fill=black, text=white, font=\large\fntTxtt]
          at (5.5,-2) (Com)
          {\begin{minipage}{16em}\color{white}
            \textcolor{myprompt}{[zr@here]\$} dvipdfmx talk.dvi
          \end{minipage}};
      \end{onlyenv}
      \begin{onlyenv}<2-3>
        \node[right, goodcode, font=\ttfamily]
          at (-5.5,2) (Src)
          {\begin{minipage}{18em}
            \myverb{\\documentclass[uplatex]\{jsarticle\}}\\
            \myverb{\\usepackage[%
              \only<2>{\alert{\bfseries dvipdfmx}}%
              \only<3>{\textcolor{mygood}{\bfseries dvipdfmx}}%
            ]\{graphicx\}}\\
            \myverb{\\usepackage\{lisp-on-tex\}}\\
            \myverb{\structure{\%...({\sffamily 以下略})...}}
          \end{minipage}};
      \end{onlyenv}
      \begin{onlyenv}<4-6>
        \node[right, badcode, font=\ttfamily]
          at (-5.5,2) (Src)
          {\begin{minipage}{18em}
            \myverb{\\documentclass[uplatex]\{jsarticle\}}\\
            \myverb{\\usepackage[%
              \only<4>{\textcolor{mybad}{\bfseries dvips}}%
              \only<5>{\textcolor{mybad}{\bfseries dvipdfm}}%
              \only<6>{\textcolor{mybad}{\bfseries dviout}}%
            ]\{graphicx\}}\\
            \myverb{\\usepackage\{lisp-on-tex\}}\\
            \myverb{\structure{\%...({\sffamily 以下略})...}}
          \end{minipage}};
      \end{onlyenv}
      \begin{onlyenv}<2>
        \node[text=alert, font=\Large]
          at (0,-0.5) (Here)
          {コレ\!};
        \draw[<-, draw=alert, ultra thick]
          (Src.center)+(-0.3,0.1)--(Here.north);
      \end{onlyenv}
      \begin{onlyenv}<3>
        \node[text=mygood, font=\Large]
          at (0,-0.5) (Here)
          {一致\!};
        \draw[<-, draw=mygood, ultra thick]
          (Src.center)+(-0.3,0.1)--(Here.north);
        \draw[<-, draw=mygood, ultra thick]
          (Com.north)--(Here.south);
      \end{onlyenv}
      \begin{onlyenv}<4-6>
        \node[text=mybad, font=\Large]
          at (0,-0.5) (Here)
          {食い違う\!\!};
        \draw[<-, draw=mybad, ultra thick]
          (Src.center)+(-0.3,0.1)--(Here.north);
        \draw[<-, draw=mybad, ultra thick]
          (Com.north)--(Here.south);
      \end{onlyenv}
    \end{tikzpicture}
  \end{mycenter}
\end{myframe}

\begin{myframe}
  \begin{mycenter}
    \begin{tikzpicture}
      \useasboundingbox[drawbb](-5.5,-4.5) rectangle (5.5,4.5);
      \begin{onlyenv}<1-2>
        \node[font=\LARGE] at (0,4)
          {\textcolor{mybad}{“ドライバ詐称”}してみよう\!};
      \end{onlyenv}
      \begin{onlyenv}<1-2>
        \node[examplecode, font=\ttfamily\scriptsize]
          at (0,0) (Src)
          {\begin{minipage}{32em}
\myverb{\% upLaTeX 文書}\\
\myverb{\\documentclass[a6paper,papersize,uplatex]\{jsarticle\}}\\
\myverb{\structure{\\usepackage[dvips]\{graphicx\}}}\\
\myverb{\\setlength\{\\fboxrule\}\{8pt\}}\\
\myverb{\\setlength\{\\fboxsep\}\{0pt\}}\\
\myverb{\\begin\{document\}}\\
\myverb{ }\\
\myverb{\\begin\{center\}\\LARGE}\\
\myverb{\{\\TeX\}ユーザの集い2014}\\
\myverb{\\par\\smallskip}\\
\myverb{\\fbox\{\structure{\\includegraphics[width=3cm]\{image.eps\}}\}}\\
\myverb{\\par\\smallskip}\\
\myverb{は\\\\\{\\gtfamily アレ。\}}\\
\myverb{\\end\{center\}}\\
\myverb{ }\\
\myverb{\\end\{document\}}
          \end{minipage}};
      \end{onlyenv}
      \begin{onlyenv}<2>
        \node[descbox, font=\Large, opacity=0.8]
          at (0,-0.6) (Desc)
          {\begin{minipage}{15em}
            \begin{itemize}
            \item ドライバ指定は%
              “\myverb{\alert{\textbf{dvips}}}”
              \par\smallskip
              \tikz{\node[badcode, font=\ttfamily\normalsize]
                {\myverb{\\usepackage[\alert{\bfseries dvips}]%
                  \{dvipdfmx\}}};}
            \item \alert{EPS画像}を挿入する
              \par\smallskip
              \begin{tikzpicture}
                \node[badcode, font=\ttfamily\normalsize, xscale=0.9]
                  at (0,0) (Part2)
                  {\myverb{\\includegraphics[width=3cm]%
                    \{\alert{image.eps}\}}};
                \node[imframe] at (0,-2.2) (Image)
                  {\pgfuseimage{texconf14eps}};
                \draw[<-, draw=black, thick]
                  (Image.east) to[out=340, in=290] (3,-0.2);
              \end{tikzpicture}
            \end{itemize}
          \end{minipage}};
      \end{onlyenv}
    \end{tikzpicture}
  \end{mycenter}
\end{myframe}

\begin{mybgcolor}{mylbad}

\begin{myframe}
  \begin{mycenter}
    \begin{tikzpicture}
      \useasboundingbox[drawbb](-5.5,-4.5) rectangle (5.5,4.5);
      \node[font=\LARGE] at (0,4)
        {\alert{古い}{\TeX}環境で処理すると…};
      \node[font=\large] at (0,3)
        {({\TeX} Live 2012)};
      \node[imframe] at (0,0) {\pgfuseimage{trial-fdn-ok}};
      \node[text=mybad, font=\hhhuge] at (-1.5,-3.5) {\facebad};
      \node[text=mybad, font=\hhuge] at (1.5,-3.5) {正常。};
    \end{tikzpicture}
  \end{mycenter}
\end{myframe}

\end{mybgcolor}

\begin{mybgcolor}{mylbanned}

\begin{myframe}
  \begin{mycenter}
    \begin{tikzpicture}
      \useasboundingbox[drawbb](-5.5,-4.5) rectangle (5.5,4.5);
      \node[font=\LARGE] at (0,4)
        {\alert{新しい}{\TeX}環境で処理すると…};
      \node[font=\large] at (0,3)
        {({\TeX} Live 2014)};
      \node[imframe] at (0,0) {\pgfuseimage{trial-fdn-ng}};
      \node[text=mybanned, font=\hhhuge] at (-1.5,-3.5) {\facebanned};
      \node[text=mybanned, font=\hhuge] at (1.5,-3.5) {破滅。};
    \end{tikzpicture}
  \end{mycenter}
\end{myframe}

\begin{myframe}
  \begin{center}\color{mybanned}
    {\hhhuge 楽しい\!\!\par}
    \myvspace{1}
    {\Large \mytanoshy\par}
  \end{center}
\end{myframe}

\end{mybgcolor}

% EOF

\subsection{その② バウンディングボックス詐称}
%
% 32fbb.tex
% 「その② バウンディングボックス詐称」

%% 画像オブジェクト
\pgfdeclareimage[width=0.9\paperwidth]%
  {trial-fbb-ok}{image/trial-fbb-ok.pdf}
\pgfdeclareimage[width=0.9\paperwidth]%
  {trial-fbb-ng}{image/trial-fbb-ng.pdf}
%
\begin{mybgcolor}{mydbad}

\begin{myframe}
  \begin{center}\LARGE\color{white}
    \mybadbkname[0.5]{②}{バウンディング\\ボックス詐称}%
        {false bounding box}
  \end{center}
\end{myframe}

\end{mybgcolor}

\begin{myframe}
  \begin{mycenter}
    \begin{tikzpicture}
      \useasboundingbox[drawbb](-5.5,-4.5) rectangle (5.5,4.5);
      \node[font=\LARGE] at (0,3.5)
        {\begin{minipage}{12\zw}\centering
          バウンディングボックス\\
          (bbox)って何
        \end{minipage}};
      \begin{onlyenv}<1>
        \node[font=\Large] at (2.9,2.9)
          {\myiswhat};
      \end{onlyenv}
      \begin{onlyenv}<2>
        \node[font=\hhhhuge, text=mybanned] at (0,0)
          {説明省略。};
        \node[font=\huge\fntKiloji, text=structure] at (0,-3)
          {そう、アレです。};
      \end{onlyenv}
    \end{tikzpicture}
  \end{mycenter}
\end{myframe}

\begin{myframe}
  \begin{mycenter}
    \begin{tikzpicture}
      \useasboundingbox[drawbb](-5.5,-4.5) rectangle (5.5,4.5);
      \begin{onlyenv}<1-3>
        \node[font=\LARGE] at (0,3.5)
          {\begin{minipage}{10\zw}\centering
            各々の画像は\alert{固有の}\\
            bboxの値を持つ。
          \end{minipage}};
      \end{onlyenv}
      \begin{onlyenv}<4>
        \node[font=\LARGE] at (0,3.5)
          {\mygood{普通の}bbox指定の方法。};
      \end{onlyenv}
      \begin{onlyenv}<5-6>
        \node[font=\LARGE] at (0,3.5)
          {画像を\>\structure{“一部だけ”}\>挿入したい!};
      \end{onlyenv}
      \begin{onlyenv}<7>
        \node[font=\LARGE] at (0,3.5)
          {\mybad{バッドな}bbox指定の方法。};
      \end{onlyenv}
      \begin{onlyenv}<1-2>
        \node[right, text=structure, font=\bfseries] at (-4.8,1.5)
          {image.eps};
        \node[imframe] at (-3,-0.5)
          {\pgfuseimage{texconf14eps}};
      \end{onlyenv}
      \begin{onlyenv}<2>
        \node[right, textbox, font=\large] at (-1.5,-0.5)
        {\begin{minipage}{12em}
          \begin{itemize}
          \item EPS形式
          \item bbox =\par
              \alert{[540 315 900 675]}
          \end{itemize}
        \end{minipage}};
      \end{onlyenv}
      \begin{onlyenv}<3-7>
        \node[right, text=structure, font=\bfseries] at (-4.8,1.5)
          {image.png};
        \node[imframe] at (-3,-0.5) (Image)
          {\pgfuseimage{texconf14png}};
        \node[right, text=black, font=\large] at (-1.5,-0.5)
        {\begin{minipage}{12em}
          \begin{itemize}
          \item \structure<3>{PNG形式}
          \item bbox =\par
              \alert{[0 0 360 360]}
          \end{itemize}
        \end{minipage}};
      \end{onlyenv}
      \begin{onlyenv}<5-6>
        \fill[white, opacity=0.75]
          (-1,-3.5) rectangle (5.5,2);
      \end{onlyenv}
      \begin{onlyenv}<5-7>
        \fill[white, opacity=0.75]
          (-1,-3.5) rectangle (-5.5,2);
      \end{onlyenv}
      \begin{onlyenv}<5-7>
        \coordinate (I1) at (Image.south west);
        \coordinate (I2) at (Image.north east);
        \path let
          \p1=($(I1)!0.05!(I2)$),
          \p2=($(I1)!0.45!(I2)$),
          \p3=($(I1)!0.95!(I2)$)
        in 
          coordinate (P1) at (\x1,\y2)
          coordinate (P2) at (\x3,\y3);
        \begin{scope}
          \clip (P1) rectangle (P2);
          \node[imframe] at (-3,-0.5)
            {\pgfuseimage{texconf14png}};
        \end{scope}
        \draw[structure, very thick, dashed]
          (P1) rectangle (P2);
        \begin{scope}[every node/.style={
          text=structure, font=\small, fill=white, opacity=0.75}]
          \begin{onlyenv}<6>
            \node[below] at (I1) {(0,0)};
            \node[above] at (I2) {(360,360)};
            \node[below] at ($(P1) + (-0.4,0)$) {(18,162)};
            \node[below] at ($(P2) + (0,-0.2)$) {(342,342)};
            \fill[structure] (I1) circle[radius=4pt];
            \fill[structure] (I2) circle[radius=4pt];
            \fill[structure] (P1) circle[radius=4pt];
            \fill[structure] (P2) circle[radius=4pt];
          \end{onlyenv}
        \end{scope}
        \begin{onlyenv}<6-7>
          \node[right, descbox, font=\large] (T2) at (-0.3,1)
            {\begin{minipage}{9em}
              \structure{“一部”} =\\
              \mybad{[18 162 342 342]}
            \end{minipage}};
          \draw[->, structure, thick]
            (T2.west) to[out=180, in=20] ($(P2)+(2pt,-1.0)$);
        \end{onlyenv}
      \end{onlyenv}
      \begin{onlyenv}<3>
        \node[text=black, font=\Large] at (0, -3.5)
          {\structure{ビットマップ画像}も同様。};
      \end{onlyenv}
      \begin{onlyenv}<4>
        \node[goodcode, font=\large, xscale=0.85]
          at (0,-3.5) (Src)
          {\myverb{\\includegraphics[\mygood{bb=0 0 360 360}]%
            \{\structure{image.png}\}}};
        \draw[->, mygood, very thick]
          (1.85,-1.4)--($(Src.north)+(1.20,-0.25)$);
        \node[text=mygood, font=\hhuge] at (2.5,-2.3) {\facegood};
        \node[text=mygood, font=\Large] at (4.0,-2.3) {普通。};
      \end{onlyenv}
      \begin{onlyenv}<7>
        \node[badcode, font=\large, xscale=0.85]
          at (0,-3.5) (Src)
          {\myverb{\\includegraphics[\mybad{bb=18 162 342 342}]%
            \{\structure{image.png}\}}};
        \draw[<->, mybad, very thick]
          (1.85,-1.4)--($(Src.north)+(1.25,-0.25)$);
        \draw[->, mybad, very thick]
          ($(T2.south)+(-1.5,0.25)$)--($(Src.north)+(-0.1,-0.25)$);
        \node[text=mybad, font=\hhuge] at (2.5,-2.3) {\facebad};
        \node[text=mybad, font=\Large] at (4.6,-2.3) {食い違う\!};
      \end{onlyenv}
    \end{tikzpicture}
  \end{mycenter}
\end{myframe}

\begin{myframe}
  \begin{mycenter}
    \begin{tikzpicture}
      \useasboundingbox[drawbb](-5.5,-4.5) rectangle (5.5,4.5);
      \node[text=black, font=\LARGE] at (0,3.5)
        {\mybad{“bbox詐称”}\:してみよう!};
      \node[textbox, font=\Large]
        at (0,0) (Desc)
        {\begin{minipage}{15em}
          \begin{itemize}
          \item ドライバは\mygood{当然}\:“dvipdfmx”。
          \item \structure{さっきの例}の通りに\\
            PNG画像を\:\alert{“一部だけ”}\:挿入。
            \par\smallskip
            \begin{tikzpicture}
              \node[badcode, font=\ttfamily\normalsize]
                at (0,0) (Part2)
                {\begin{minipage}{16em}
                  \myverb{\\includegraphics}\\
                  \myverb{\mbox{~~}[\alert{bb=18 162 342 342}}]%
                  \{image.png\}
                 \end{minipage}};
            \end{tikzpicture}
          \end{itemize}
        \end{minipage}};
      %
      \coordinate (I0) at (0,-4);
      \node[imframe, opacity=0.4] at (I0) (Image)
        {\pgfuseimage{texconf14png}};
      \coordinate (I1) at (Image.south west);
      \coordinate (I2) at (Image.north east);
      \path let
        \p1=($(I1)!0.05!(I2)$),
        \p2=($(I1)!0.45!(I2)$),
        \p3=($(I1)!0.95!(I2)$)
      in 
        coordinate (P1) at (\x1,\y2)
        coordinate (P2) at (\x3,\y3);
      \begin{scope}
        \clip (P1) rectangle (P2);
        \node[imframe] at (I0)
          {\pgfuseimage{texconf14png}};
      \end{scope}
      \draw[structure, very thick, dashed]
        (P1) rectangle (P2);
    \end{tikzpicture}
  \end{mycenter}
\end{myframe}

\begin{mybgcolor}{mylbad}

\begin{myframe}
  \begin{mycenter}
    \begin{tikzpicture}
      \useasboundingbox[drawbb](-5.5,-4.5) rectangle (5.5,4.5);
      \node[font=\LARGE] at (0,4)
        {\alert{古い}{\TeX}環境では…};
      \node[font=\large] at (0,3)
        {({\TeX} Live 2012)};
      \node[imframe] at (0,-1) {\pgfuseimage{trial-fbb-ok}};
      \node[text=mybad, font=\hhhuge] at (4.5,2.8) {\facebad};
    \end{tikzpicture}
  \end{mycenter}
\end{myframe}

\end{mybgcolor}

\begin{mybgcolor}{mylbanned}

\begin{myframe}
  \begin{mycenter}
    \begin{tikzpicture}
      \useasboundingbox[drawbb](-5.5,-4.5) rectangle (5.5,4.5);
      \node[font=\LARGE] at (0,4)
        {\alert{新しい}{\TeX}環境では…};
      \node[font=\large] at (0,3)
        {({\TeX} Live 2014)};
      \node[imframe] at (0,-1) {\pgfuseimage{trial-fbb-ng}};
      \node[text=mybanned, font=\hhhuge] at (4.5,2.8) {\facebanned};
    \end{tikzpicture}
  \end{mycenter}
\end{myframe}

\begin{myframe}
  \begin{center}\color{mybanned}
    {\hhhuge 楽しい\!\!\!\par}
    \myvspace{1}
    {\Large \mytanoshy\par}
  \end{center}
\end{myframe}

\end{mybgcolor}

% EOF

\subsection{その③ 面倒なのでピクセル単位}
%
% 33lip.tex
% 「その③ 面倒なのでピクセル単位」
%

%% 画像オブジェクト
\pgfdeclareimage[width=0.5\paperwidth]%
  {trial-lip-ok}{image/trial-lip-ok.pdf}
\pgfdeclareimage[width=0.5\paperwidth]%
  {trial-lip-ng}{image/trial-lip-ng.pdf}

\begin{mygroup}
\setbeamercolor{background canvas}{bg=mydbad}
\begin{frame}[plain]{}
  \begin{center}\LARGE\color{white}
    \mybadbkname[0.5]{③}{面倒なので\\ポイント単位}%
        {lazily-in-point}
  \end{center}
\end{frame}
\end{mygroup}

\begin{mybgcolor}{mylbad}

\begin{myframe}
  \begin{mycenter}
    \begin{tikzpicture}
      \useasboundingbox[drawbb](-5.5,-4.5) rectangle (5.5,4.5);
      \node[font=\LARGE] at (0,4)
        {\alert{大昔の}{\TeX}環境では…};
      \node[font=\large] at (0,3)
        {(2005年頃の{\TeX}環境)};
      \node[imframe] at (0,-0.3) {\pgfuseimage{trial-lip-ok}};
      \node[text=mybad, font=\hhhuge] at (-1.5,-4) {\facebad};
      \node[text=mybad, font=\hhuge] at (1.5,-4) {正常。};
    \end{tikzpicture}
  \end{mycenter}
\end{myframe}

\begin{myframe}
  \begin{mycenter}
    \begin{tikzpicture}
      \useasboundingbox[drawbb](-5.5,-4.5) rectangle (5.5,4.5);
      \node[font=\LARGE] at (0,4)
        {\alert{新しい}{\TeX}環境では…};
      \node[font=\large] at (0,3)
        {({\TeX} Live 2014)};
      \node[imframe] at (0,-0.3) {\pgfuseimage{trial-lip-ok}};
      \node[text=mybad, font=\hhhuge] at (-1.5,-4) {\facebad};
      \node[text=mybad, font=\hhuge] at (1.5,-4) {正常\!};
    \end{tikzpicture}
  \end{mycenter}
\end{myframe}

\end{mybgcolor}

\begin{mybgcolor}{mylbanned}

\begin{myframe}
  \begin{mycenter}
    \begin{tikzpicture}
      \useasboundingbox[drawbb](-5.5,-4.5) rectangle (5.5,4.5);
      \node[font=\LARGE] at (0,4)
        {\alert{少し古い}{\TeX}環境では…};
      \node[font=\large] at (0,3)
        {({\TeX} Live 2012)};
      \node[imframe] at (0,-0.3) {\pgfuseimage{trial-lip-ng}};
      \node[text=mybanned, font=\hhhuge] at (-1.5,-4) {\facebanned};
      \node[text=mybanned, font=\hhuge] at (1.5,-4) {破滅。};
    \end{tikzpicture}
  \end{mycenter}
\end{myframe}

\begin{myframe}
  \begin{center}\color{mybanned}
    {\hhhuge 楽しい…{}?\par}
    \myvspace{1}
    {\Large \mytanoshy\par}
  \end{center}
\end{myframe}

\end{mybgcolor}

\begin{mybgcolor}{mydbanned}

\begin{myframe}
  \begin{center}\color{white}
    {\hhhhuge\bfseries 楽しくない\!\!\par}
    \myvspace{1}
    {\gigantic \facebanned\par}
  \end{center}
\end{myframe}

\end{mybgcolor}

\begin{mybgcolor}{mylgood}

\begin{myframe}
  \begin{mycenter}
    \begin{tikzpicture}
      \useasboundingbox[drawbb](-5.5,-4.5) rectangle (5.5,4.5);
      \node[text=black, font=\hhuge\bfseries, xscale=0.95]
        at (0,3) {\mygood{正しい}コードを書こう!};
      \node[text=mygood, font=\gigantic] at (0,0) {\facegood};
      \node[text=mygood, font=\fntVMin\LARGE, rotate=-90]
        at (-1.9,-2.4) {まさに};
      \node[text=mygood, font=\fntVMin\hhuge, rotate=-90]
        at (-2.7,-3.0) {正論。};
    \end{tikzpicture}
  \end{mycenter}
\end{myframe}

\end{mybgcolor}

% EOF

\section{正しさのすすめ}
%
% 40justice.tex
% 「正しさのすすめ」
%
% EOF

\section{御清聴アレ}
%
% 40justice.tex
% 「正しさのすすめ」
%

\begin{mybgcolor}{mylgood}
\begin{myframe}
  \begin{mycenter}
    \begin{tikzpicture}
      \useasboundingbox[drawbb](-5.5,-4.5) rectangle (5.5,4.5);
      \node[text=mygood, font=\hhuge] at (-5,-4) {\facegood};
      \node[text=mygood, font=\hhuge] at (+5,-4) {\facegood};
      \node[text=mygood, font=\hhuge] at (-5,+4) {\facegood};
      \node[text=mygood, font=\hhuge] at (+5,+4) {\facegood};
      \node[right, text=mygood, font=\rmfamily\hhhhhhuge]
        at (-5,1.5) {Happy};
      \node[left, text=mygood, font=\rmfamily\hhhhhhuge]
        at (4.6,-1.5) {{\TeX}ing};
      \node[right, text=mygood, font=\rmfamily\hhhhhhuge,
        xslant=0.3, yscale=1.2]
        at (4.5,-1.4) {!\strut};
    \end{tikzpicture}
  \end{mycenter}
\end{myframe}
\end{mybgcolor}

% EOF


\end{document}
% EOF
\iffalse%

◇
dvipdfmxと
3つのバッド・ノウハウ

◇
この物語《LT》は
LaTeX + dvipdfmxでの
画像の挿入に関する
封印された
バッドノウハウの
記録である。

◇
バッド・ノウハウ
is
何

◇
このLTにおいて!
バッド・ノウハウ
is
バグ技

◇
正攻法       バッド・ノウハウ
正しいコード    /間違ったコード/
↓              
欲しい出力      (仕様上結果は未定義)
;-)
〈つまらない〉

is
バグ技

◇
正攻法       バッド・ノウハウ
正しいコード    /間違ったコード/
↓              ↓偶然
欲しい出力      欲しい出力
;-)             8-P イイネ!

バージョンアップ!

◇
正攻法       バッド・ノウハウ
正しいコード    /間違ったコード/
↓              ↓偶然
欲しい出力      .間'違:っ:た 出  力
;-)             X-O ウワァァァァ

◇
バッドノウハウ
その①
ドライバ詐称
false driver name

◇
\usepackage[dvipdfmx]{graphicx}
;-)

[zr@texconf14]$ talk.dvi


◇
\usepackage[dvipdfm]{graphicx}
8-P

[zr@texconf14]$ dvipdfmx talk.dvi

◇
\usepackage[dvips]{graphicx}
\usepackage[dviout]{graphicx}


◇
(ソース①)

◇
TeX Live 2012
8-P

◇
TeX Live 2013
X-O

◇
楽しい!!!

◇
バッドノウハウ
その②
バウンディングボックス詐称
false bounding box

◇
画像ファイルは固有の
“バウンディングボックス”(bbox)
を持っている。

texconf.pdf

bboxの値
[100 100 500 500]

◇
画像ファイルは固有の
“バウンディングボックス”(bbox)
を持っている。

texconf.png

bboxの値
[0 0 400 400]


◇
(extractbb しない場合)
正しいbboxの値の指定が必要。

\includegraphics[
  bb=0 0 400 400,
  width=6cm]{texconf.png}
:-)

◇
間違ったbboxの値の指定。
(8-P「画像の一部を切り出したい」)

\includegraphics[
  bb=200 300 300 400,
  width=6cm]{texconf.png}
8-P

◇
TeX Live 2012
8-P

◇
TeX Live 最新
X-O

◇
楽しい!!!

◇
バッドノウハウ
その③
面倒なのでピクセル単位
lazily-in-pixels

◇
texconf.png
96dpi
400x400

正しいbboxの値は?


◇
texconf.png
96dpi
400x400

解像度は 96dpi
∴ 96 px = 1 in = 72 bp
∴ 400 px = 300bp

◇
texconf.png
96dpi
400x400

bb=[0 0 300 300]
:-)

◇
texconf.png
96dpi
400x400

bb=[0 0 400 400]
8-P

◇
2006年頃
8-P

◇
最新
X-O

◇
楽しくない!
X-O

◇
正しいコードを書こう!
;-)
まさに正論

◇
dvipdfm


◇
dvipdfm
は 2013年7月を以て
終了
致しました。
:-) Thank you & Good bye! :-)

◇
ステップ0
dvipdfm ×
dvipdfmx
を使う。

◇
ステップ①
(現状の TeX Live では)
extractbb の自動起動を有効にする。
その方法は…

◇
TeX Wiki 見ろ。
以上。

◇
ステップ③
dvipdfmx ドライバを指定する

\usepackage[dvipdfmx]{graphicx}
\usepackage[dvipdfmx]{color}

◇
ステップ④
\includegraphics する。
bb は付けない。

\includegraphics[width=6cm]{texconf.png}

◇
ステップ⑤
結果。

〈なんと画像がちゃんと入ります。すごいですね。〉

◇
楽しい

◇
Happy
   TeXing!
〈以上です〉

\fi%
