%
% 10title.tex
% 「タイトル」
%
\begin{mybgcolor}{mydbad}

\begin{myframe}\color{white}%
  \begin{mycenter}
    \begin{tikzpicture}
        [every node/.style={font=\rmfamily}]
      \node at(-1.5,5.0) {\color{mygood!40!white}\hhhhuge dvipdfmx};
      \node at (0,3.7) {\color{white}\Large と};
      \node at (0,2.5) {\color{white}{\hhhuge 3}{\Large つ}};
      \node at (0,1.3) {\color{white}\Large の};
      \node at (1,0) {\color{mybad!25!white}\hhhuge バッド・ノウハウ};
    \end{tikzpicture}
    \myvspace{0.5}
    {{\large 八登\ 崇之}\quad(Takayuki YATO)}\par
    \smallskip
    {{\small\textcolor{black!25}{Twitter ID:}}
      \textcolor{yellow!75!white}{@zr\_tex8r}}\par
    \smallskip
    {\small{\TeX}ユーザの集い2014\qquad 2014年11月8日}
  \end{mycenter}
\end{myframe}

\newcommand*{\myemph}[1]{\textcolor{mylbad}{\bfseries #1}}
\begin{myframe}\color{white}
  \begin{mycenter}\large
  \begin{minipage}{15em}
  この物語{\small (LT)}は
  \par\myvspace{0.75}\hspace*{1em}
  \myemph{dvipdfmx}で
  \par\myvspace{0.75}\hspace*{2em}
  \myemph{画像を挿入}するときの
  \par\myvspace{0.75}\hspace*{3em}
  {\LARGE\bfseries\textcolor{mybanned!50!white}{“封印”}}された
  \par\myvspace{0.75}\hspace*{4em}
  \myemph{バッドノウハウ}を
  \par\myvspace{0.75}\hspace*{5em}
  記したものである。
  \end{minipage}\end{mycenter}
\end{myframe}

\end{mybgcolor}

% EOF
