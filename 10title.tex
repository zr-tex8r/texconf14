%
% 10title.tex
% 「タイトル」
%
\begin{mygroup}
\setbeamercolor{background canvas}{bg=mydbad}
\begin{frame}[plain]{}\color{white}%
  \begin{center}
    \myhac
    \begin{tikzpicture}
        [every node/.style={font=\rmfamily}]
      \node at(-1.5,5.0) {\color{mygood!40!white}\hhhhuge dvipdfmx};
      \node at (0,3.7) {\color{white}\Large と};
      \node at (0,2.5) {\color{white}{\hhhuge 3}{\Large つの}};
      \node at (0,1.3) {\color{white}\Large の};
      \node at (1,0) {\color{mybad!25!white}\hhhuge バッド・ノウハウ};
    \end{tikzpicture}
    \myvspace{1}
    {{\large 八登\ 崇之}\quad(Takayuki YATO)}\par
    \smallskip
    {\small{\TeX}ユーザの集い2014\qquad 2014年11月8日}
    aa
  \end{center}
\end{frame}

\begin{frame}[plain]{}\color{white}\large%
  \begin{center}
  \begin{minipage}{15em}
  この物語{\small (LT)}は
  \par\myvspace{0.75}\hspace*{1em}
  \textcolor{mylbad}{dvipdfmx}で
  \par\myvspace{0.75}\hspace*{2em}
  \textcolor{mylbad}{画像を挿入}するときの
  \par\myvspace{0.75}\hspace*{3em}
  {\LARGE\bfseries\textcolor{myban!50!white}{“封印”}}された
  \par\myvspace{0.75}\hspace*{4em}
  {\bfseries\textcolor{mylbad}{バッドノウハウ}}を
  \par\myvspace{0.75}\hspace*{5em}
  記したものである。
  \end{minipage}\end{center}
\end{frame}

\end{mygroup}

% EOF
